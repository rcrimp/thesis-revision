% \begin{center}
    
% Information is valuable.

% Information can be encoded with language.

% Digitised language can be preserved and shared infinitely.

% Information Retrieval mediates our access to Digital Information.

% The Language Barrier separates us from Information Retrieval Systems.

% A symptom of the Language Barrier is Vocabulary Mismatch.

% Vocabulary Mismatch causes Retrieval failure.

% Retrieval failure makes me sad.
    
% \end{center}

% \vspace{\baselineskip}
Information is one of the most valuable resources we have, and the primary way we access information nowadays is via search engines. Unfortunately, written language is rife with inconsistencies and ambiguity leading to a problem known as Vocabulary Mismatch, which is when authors use different words to describe the same thing. In a text retrieval search engine, mismatch can be addressed with Semantic Query Expansion: adding more words to a user's search query, such that it contains a wider vocabulary. If expansion terms are not chosen carefully \textit{query drift} can occur, which is where the query's semantics drifts away from what the user meant. This thesis proposes two methods to mitigate query drift directly. The first, Term Frequency Merging (Chapter 7), modifies the ranking function by accumulating the term frequency of each expansion term with its respective original query term, this prevents TF*IDF ranking functions (like BM25) from over boosting words that have disproportionately more expansions than other words. The second, Query Context Selection (Chapter 8), is an expansion term selection process which prevents the inclusion of expansions terms with contradictory semantics. This is done by using the semantic context shared by the original query terms and using that context to select the expansions which most strongly relate to the original query as a whole. The results from both experiments are promising, they both out perorm no expansion and na{\"i}ve expansion, especially prominent improvement when query drift has caused the most damage. However, other query reformulation approaches like Blind Relevance Feedback still outperform the two proposed methods in many test cases.