\chapter{Language Barrier}

\section{Meaning in Communication}
In speech an utterance is spoken by one then heard by another. We won't consider the lone speaker in a forest.

Speaker
intended meaning = literal semantics, contextual implicature.

Listener
interpreted meaning = literal semantics, contextual inference.

understanding is when intended meaning matches interpreted meaning

literal semantics could differ, through ambiguities.
implicature and inference could differ, through misunderstanding of assumed context.

Some nonsense sentences have neither intended or interpreted meaning.
``Work is the curse of the drinking class''
did he mean
* only the drinking class, work
* only the drinking class feel work to be a curse
* those who do not drink, are not cursed with work
 
There was likely no clear intended meaning, he only cared that you laughed at his witty recontextualization. I think it's likely that hear did not care about the exact inferred meaning of the quip, so long as you infer something worth admiring, he was a supremely self-involve man.


\section{Informational Context}
Informational Stance



\section{Atomist}
atomist vs holistic

individual words can be ambiguous and vague
phrasal verbs can be ambiguous and vague
grammatical structures can be ambiguous and vague
logical structures can be ambiguous and vague
higher level social concepts like ownership are ambiguous and vague
absurdities and nonsense are also ambiguous and vague