\chapter{The Miraculous Gift of Language}

The English poet Alexander Pope wrote \textit{``To err is Human''}... because animals and machines cannot commit errors, animals behave only according to their biological-programming, and machines to their algorithmical-programming. Humans are the only entity in the known universe that can make mistakes, as only humans experience their conscience that lets them determine \textit{right} from \textit{wrong}.

The English poet W. H. Auden wrote \textit{``To speak is human''}... because animals and machines cannot engage in speech, animals express only their base level desires according to their biological-programming, and machines can only express what their algorithmical-programming dictates. Humans are the only entity in the known universe that can converse, as only humans experience their self-consciousness that lets them distinguish \textit{I} from \textit{Other}. 

If language and errors are necessarily human, then errors in language are most humane.

I do not sincerely support the opinions in the previous two paragraphs. It's thought provoking to consider what exactly is an error in judgement. 

Clearly animals make errors in judgement all the time. I've seen cat's misjudge leaps

And prior to 1997 the best AI developed to play chess

Prior to Deep Blue beating Kasparov (the international world Chess champion) in 1997. any algorithm programmed to play chess made blunders despite being a deterministic machine.

Machines and animals can also communicate, but at a far simpler level. 

\section{Anthropocentric Teleology}
Speech is the noise of thought.

It exists only to communicate our subjective experience.

That is to say, language is a phenomena that exists for a very specific and definable purpose, a purpose constructed by society; i.e.\ language is extrinsically teleological. 

\noindent


\begin{center}
\textit{Artistic expression
\\Deepest desires
\\Darkest secrets
\\Beautiful lyrics
\\Scientific facts
\\Innovative ideas}
\end{center}

Speech can represent worldly knowledge, it can also change the world.

\begin{center}
\textit{Speak Truthly
\\Lie falsely
\\Proclaim love
\\Confess a crime
\\Upset gods with blasphemy}
\end{center}




\section{Jokes}
%I'm Hungry... Hi Hungry, I'm Dad.
%ambiguity of the verb to be, which doesn't work in french where you say that you have hunger. But you can still say "I am sorry"... you do not have sorrow.

The best way to explain languages mistakes are with jokes. Firstly statements are more memorable when they're entertaining, but most importantly, all jokes rely on the deliberate exploitation of language mistakes. Not everything that is funny is a joke, witnessing your best friend fall over after being kicked in the balls is subjectively hilarious, but it's not a \textit{joke-joke}, perhaps a cruel practical joke that exploits the invalidity of some non-linguistic rule. Herein I am considering only linguistic-jokes expressed in \textit{verbal humour}.

\section{Conversational Humour}
One attempt to create a unified theory of jokes in conversation is the deliberate violation of the Cooperative Principle; flout one of Grice's conversational maxims.

\begin{center}
    Alice: \textit{Can you call me a taxi?}
    \\ Irene: \textit{OK, you're a taxi!}
\end{center}

Here the ambiguity of the verb \textit{to call} is exploited for comedic effect. If the English language was without mistakes, this could not have happened. Here's another:

\begin{center}
    Alice: \textit{Do you have the time?}
    \\ Irene: \textit{Yes.}
\end{center}

Again, Irene deliberately misunderstands the question, there is an obvious Speech Act, an indirect demand: \textit{``tell me the time''}; she instead chooses to answer the literal interpretation of the question. We call this behaviour \textit{Playing Dumb, Feigning Ignorance, Selective Obliviousness, Playing Possum}, and being \textit{Deliberately Obtuse}. Here is another example of Irene acting in a uncooperative manner.

\begin{center}
    Waiter: \textit{Are you here for dinner?} 
    \\ Irene: \textit{No, I am here to steal your plates.}
\end{center}

The waiter's could be expressing a formality of the restaurant setting, or perhaps they're attempting to determine whether Irene wants a meal or just a drink so as to choose an appropriate seating location for her. Either way, Irene chooses to admit her desire for criminal activity instead! an inappropriate admission, clearly misguided or an outrageous lie. A dull waiter might be fooled by her sharp wit. Ideally however, they'd acknowledge the absurdity of the crockery crime confession and laugh in surprise!

%Irene's utterance is a case of \textit{verbal irony}, where the intended meaning is contrary to the literal semantics. Verbal irony can be identified as a joke in two primary ways, First is absurdity, an extreme disparity from reality makes the proposition impossible, illogical or unlikely. The second is if the utterance is delivered with a sarcastic tone. Tone is difficult to directly convey in text, but Internet culture developed a format of alternating case 

%May 5th 2017



\section{Manufacture a Conversational Joke}
From my understanding of conversational humour I have built a model language comprehension model which includes a step that diverges into the wonderful world of whimsy!

\begin{enumerate}

    \item \textbf{Infer} the speakers intended interpretation.
    \item \textbf{Discover} an incidental or \textbf{Construct} an alternative interpretation.
    \item \textbf{Deliver} a response for the the second interpretation.
    \item \textbf{???}
    \item \textbf{Profit.}

\end{enumerate}

There are clearly more steps, like considering whether joking is appropriate in the current social environment, which as I have discovered first-hand is rarely the case ...for me; someone else more cowardly may choose to bite their tongue during a job interviews, police inquires, airport security checks, or when writing an academic thesis. 

The magic happens in step 2. This step requires an exceptionally sharp or an exceedingly dull mind. It is step 3 that flouts the Cooperative Principle, which ordinarily, would be a request to clarify the speakers original statement. This level of wit shows a deep understanding of the conversation game, and proves the speakers high level of verbal fluency as a conversationalist, one of the highest virtues in high society. I personally would never claim to be a good conversationalist, though I do regularly prove it. These conversational jokes are crucial to certain language games we play with friends, like banter, flirting and shit-talk. The other principle of good conversation I often see ignored is balanced turn taking. Both academics and comedians are burdened with the delusion that their opinions want to be heard, so they tend to not to listen and interrupt frequently. Many conversational bullies in these social circles.

\section{Scripted Jokes}
In scripted comedy the joke takes on both sides of the \textit{conversational joke}, in what is widely known as the \textit{setup-punchline} format. The setup establishes a premise with at least two interpretations, one should be strikingly obvious, the other non-obvious. The punchline invalidates the obvious interpretation in favor of the non-obvious. This revelation is when we laugh. 

\begin{center}
\textit{I thought I was allergic to camels when I discovered a rash on my inner-thigh
\\ apparently that's the wrong way to smoke them.}
\end{center}

subvert expectations, bait n switch
The reveal, the turn, the twist, 

This kind of misdirection is known in linguistic as a \textit{forced reinterpretation}, and it is known in comedy circles as a \textit{pull back and reveal}. 

The joke-writer usually achieves this by intentionally exploiting an ambiguity, vagueness or other mistake in language to mislead the listener into making an assumption, which can be later invalidated. 



The most critically acclaimed comedian in Britain describes the prevalence of these jokes in stand-up comedy.

\begin{center}
\textit{At a rough estimate, half of what we find amusing involves using little linguistic tricks to conceal the subject of our sentences until the last possible moment, so that it appears we are talking about something else}\footnote{https://www.theguardian.com/world/2006/may/23/germany.features11}
\\ --- Stewart Lee
\end{center}

%forced reintepreation includes intentional misunderstand for comedic effect.

This definition nicely covers all linguistic jokes of the setup-punchline format, but like the conversational theory it fails to account for \textit{anti-jokes}, \textit{in-jokes}, the \textit{implicature of innuendo}, \textit{verbal irony} inflected with sarcastic tone, or even the \textit{uncomfortable truths} which are typical of a dark sense of humour. These other kinds of jokes rely on a deeper level of semantics that are difficult to elucidate, they were hard enough to even categorise. We will not be exploring deep jokes.

The next chapter is a series of jokes, followed by an explanation into the linguistic mistake, and then an exploration into the wider implications of the miscommunication. A frog famously once said ``explaining a joke is like dissecting the American author E.B. White, we learn how it works but E.B. White dies, preferably before writing Stuart Little''. I may have misremembered that. I suggest you read the jokes slowly, do not speed read them to obtain their informational content, for they are to be experienced, not read. Though, if \textit{performed} as intended they would be funnier.

elucidate a deeper meaning

connecting together a variety of ideas across multiple areas

Something which is undeniable, but not immediately obvious

%I have personally verified all the jokes to be subjectively funny, you can deny that they're funny to you, but you can't deny their inherent linguistic structure as jokes. 
%To get the full effect of these jokes in their written form, that is unperformed, 


Language is a tool for communicating ideas (emotion, facts, ...)
Language is not only the tool, but also the structure for which the information is inside of.








\section{Rules of Language}
Language is governed by rules. Rules that are described, ascribed then prescribed. Rules which are explicitly stated and followed, rules unstated and tacitly followed, and sometimes unstated and unconsciously followed\footnote{An unknown known.}. Like the explicit rules of regular verbs and the implicit rules of irregular verbs. The irregularities of language-rules are most prominent when a child blunders in speech, as the the irregularities of social-rules are when they say the Darndest Things.

\begin{center}
\textit{``I goed more faster''
\\``You're a grown up, I'm a grown down''
\\``My mum looks beautifully, your mum looks uglily''
}
\end{center}

\section{Mistakes and Errors}

language is a game learned through mimicry
Errors in speech vs errors in language itself.

Define miscommuniacation
Define language mistakes

Miscommunication is problematic.

The set of rules are called a grammar

linguistic ignorance is an excuse

knowledge theory - epistimology
ignorance

Mistake - Accidental mispeak.
Error - A mistake with connotation of a technical nature or seriousness.
Fault - A mistake with connoations of blame.
Blunder - A mistake with connotations of carelessness.

Wrong - A mistake with connotations of ignorance, the mistake can only be identified by another.

