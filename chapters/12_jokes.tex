\chapter{The Jokes}
\section{Ambiguity and Vagueness}
First let's get the obvious and easy jokes out of the way

\subsection{Polysemic Ambiguity}

Homophone, homograph, 

%In general speaking more vaguely or ambiguously in conversation is annoying, h

\textit{``Every 8 seconds somewhere in the United States there is a woman giving birth... we need to find her, and stop her''}

"a woman" is ambiguous as to the denotation of "any one woman" or the referent "a specific woman". 

This ambiguity is why it's hard to differentiate between denotation and reference.

literal figurative ambiguity

\begin{center}
\textit{``If you are in the market for easy laughs, you learn that two well-tried ways are either to trip up a cliche or take things absolutely literally."}
\\ --- Terry Pratchett
\end{center}

\subsection{Phrasal Ambiguity}
Sometimes the ambiguity is in not found within a single word itself, but rather in the larger phrase.

\begin{center}
    \textit{``Linguists enjoy ambiguity more than most people''}
\end{center} 

\subsection{Prepositional Ambiguity}

\begin{center}
	\textit{``I shot an elephant in my pyjamas, how he got in my pyjamas I'll never know''}
	%\textit{``I know a man with a wooden leg named Smith. So I asked him "What was the name of his other leg?"}
	--- Groucho Marx
\end{center} 

Prepositions

location
state
posession
time

More interesting, and more difficult to notice are ambiguities within syntax, where the individual words are attributed with the same senses, but two (or more) distinct syntactic structures can be assigned.

The use of syntactically concealed ambiguities are often used in wordplay jokes, as above where Groucho uses prepositional phrase attachment ambiguity to conceal the fact that it was in fact not he who was wearing the pyjamas. Grammatical ambiguity can arise in many contexts which is an issue when clarity is important, e.g.\ legal documents, search queries etc... The following sentence has at least five distinct syntactic structures, can you find all five?. 

\begin{center}
	\textit{``I saw a man on a hill with a telescope.''}
\end{center}

\subsection{Adjective Ambiguity}

Another example I don't know the name for is the following

\begin{center}
	\textit{``A big ass television.''}
\end{center}

\begin{center}
	\textit{``A big-ass television.''\\ ``A big, ass television.''}
\end{center}

\begin{center}
	\textit{``A crazy rich asians.''}
\end{center}

\begin{center}
	\textit{``Comedy Sex God''}
\end{center}

Either a television that is both big and old... or possibly it's the informal phrase ``Big old" with the emphatic meaning ``very big"... I don't understand the rules of adjectives as well as I do prepositions.

Say something about Noam Chomsky's opinions on grammar.

\subsection{Verb Ambiguity}

I hate it when my foot falls asleep during the day because that means it's going to be up all night - Steven Wright

the disparity between figurative language and literal language.

To fall asleep as a standard definition, but also an idiomatic defintion.

To sleep with the fishes


the most ubiquitous form of polysemy is caused by the usage of phrasal verbs. For example the words ``set'' and ``run'' are commonly used in phrasal verbs, and as such have hundreds of distinct senses in contemporary dictionaries. 

\begin{center}
	\textit{to set out, to set up, so set down, to set forth...}

    \textit{to run out, to run up, to run down, to run in...}
\end{center}

If you were looking for information on 
Grow my hair out

In a more strict literal sense

Grow out (to it's maximal length)
(fall) out

We (as humans) recognise phrasal verbs from each other, through repetitive usage, 

as the first interpretation, but a machine by default has no background of experience to draw from

%\subsection*{Dialects}
There are also many examples which are even more subtle, cultural dialects that even humans often stumble upon. Many of which have yet to be recorded officially in dictionaries. 

``I have one house rule: If you're in my home, take your thongs off'' 

A completely innocuous statement in Australian-English, but in American-English it would raise a few eyebrows.
















\section{Gender Ambiguity}
\begin{center}
\textit{``I wish I was a woman, because I'm deathly afraid of man-eating tigers''} \\ --- Daniel Crisp
\end{center}

issues of gender in language, Sapir-Worf hyphothesis

Using \textit{``man''} as a suffix or prefix is considered by many to be misogynistic here in 2019. And there it is certainly a flaw that language has words like \textit{mankind}, which implicitly exclude \textit{woman}. The denotation \textit{``Fireman''} and \textit{``Policeman''} imply the exclusion of women, and the referents \textit{``Fireman''} and \textit{``Policeman''} imply a distinction between genders in these roles. Should we instead employ gender neutral language \textit{``Fire Fighter''} or \textit{``Police Officer''}? Yes.

I think the evolution of gender neutral language is good for a couple of reasons. Firstly, I would like to disagree with the mainstream social critique, which suggests that the suffix \textit{-man} is sufficient evidence for sublimated sexism in language. It's folk etymology at it's worst. The suffix actually derives from the vocabulary related to working and \textit{manual labour}, which derives from the Latin \textit{``Manus"} for hand. Whereas the male human \textit{``man''}, comes from the Germanic \textit{"Mann"}, an evolutionary coincidence of linguistic form. I will concede that the lexical form of words would make it easier for a misogynist to communicate misogyny, but the words themselves are harmless.

My first reason for supporting gender neutral language: It's an interminable task to learn the gender assignment of nouns in both German and French, and from where I stand it's quintessentially useless information! except for crass sexist humour\footnote{Le ``compl\'{e}ment'' est masculine et la ``critique'' est f\'{e}minin... pour des raisons \'{e}videntes.}. And it's especially frustrating learning both these languages at the same time, as the gender assignments often contradict each-other.

My second reason for supporting gender neutral language: If the \textit{-man} prefix becomes antiquated or a vestigial part of language, my ``man-eating tiger'' joke will die. Is removing the androcentric words from language required to end misogyny in society? It wouldn't make anyone more sexist. As a comedian I love gender neutral words, because ambiguous words make it easier to write jokes. As a linguist I enjoy ambiguity more than most people.

\textit{Jim and Lisa got married, \textbf{she} baked the cake.}

\textit{Jim and Lisa got married, \textbf{they} baked the cake.}

They statement is ambiguous. Did Jim or Lisa make the cake? or the plural form of they indicating they both made the cake. We can remove the plural ambiguity like this:

\textit{Jim and Lisa got married, \textbf{they} baked the cake themself.}
 
Obviously this is just a contrived example and isn't a problem in speech, as you you can  replace ambiguous pronouns with the actual name.

%I now force myself to say ``they'' in conversation, so that if I ever arrive upon a situation where I can crack a joke simply by saying ``he'' when my conversational partner believes I'm talking about a woman. 

\section{Negation}

A great joke from the 1939 movie Ninotchka

\textit{``a man asks the barrista for coffee without cream, the waiter says I'm sorry sir, we run out of cream, but we still have milk, would you instead like a coffee without milk?"}

Hegel the German Idealist
Determinate Negation
Phenomenology of Spirit

"plain coffee"

"coffee without cream"

"coffee without milk" 

are all different concepts

the negated milk is a different concept to that of negated creme

the sensory experience of the coffee remains the same, the material objects are indistinguishable, but the meaning of the wider context is ideally different 

like expectations which are experiential factors

our mental representation is markedly different

things are defined but what they are not


\section{Nothing is Something}

In John 8:1–11, Christ says to those who want to stone the woman accused of adultery, \textit{``Let him who is without sin among you be the first to throw a stone!''} he is immediately hit by a stone, and then shouts back: \textit{``Mother! I asked you to stay at home!''}

The joke lies within the phrase \textit{``him who is without sin''} assumed to be \textit{``nobody''} for according to Christian doctrine everybody sins. The punchline subverts this definition by alluding to Mary, mother of Jesus, an archetype of holiness, she being the only known case of mammalian parthenogenesis, having never discovered carnal knowledge. Within Catholic doctrine virginity is a requirement for sinless purity.

\textit{``Before bedtime the programmer puts two glasses on his bedside table. One full of water in case he gets thirsty, and one empty in case he does not."}

Jean-Paul Sartre the French Existentialist

Being and Nothingness

the absence of something, or the negation of something is still something. 

%If we had a magic spell that could remove milk from any drink coffee with milk would become coffee without milk, but a coffee with creme would remain coffee with cream. 

If we consider what the customer desires as a result

The customer will get some amount coffee, hopefully served in a drinking vessel, and with a indeterminate amount of sugar.

Information retrieval exists in a very specific context which is similar to the customer wanting coffee. Precisely defining the pragmatics of this context has not been attempted because we rely to heavily on our intuitive understanding. Everyday we are involved in these kinds of operations.

what the customer wants, we need to extrapolate the intentions behind the words, which is a different problem than determining the meaning precisely and unambiguously.




\section{Double Negative}

Bob: \textit{``Would you like to come upstairs for some coffee?''}

Alice: \textit{``sorry, I do not drink coffee.}

Bob: \textit{That's OK, I don't event have coffee ;)}

proposition: proposal to drink coffee is suggested

first negation: the proposal is rejected

second negation: the initial proposition is rejected not back to the original proposition, but to a third alternative.

all three proposition mean sex

Let's drink coffee together (sex)

Let's not drink coffee (sex)

Let's not drink coffee (sex)

The colloquial grammar rules of negation in spoken English is that the double negative is the identity, which is logically unsound. Some urban English dialects employ emphatic negation, where extra negative terms instead reinforce the original negation.

In primary school arithmetic we are taught that man forms of negation are identity $5 = -1 * -1 * 5 = (5^{-1})^{-1}$. 

And also classical logic where $a = !(!a)$.

But these are not the rules that govern negation in semantics broadly.

double negatives cancel one another and produce an affirmative

\subsection{Double Affirmative}

During a lecture the Oxford linguistic philosopher J. L. Austin claimed that while double negatives implies a positive in English, there is no such language where a double positive implies a negative. To which his student, Sidney Morgenbesser responded in a dismissive tone, \textit{``Yeah, yeah..."} (often quoted as \textit{``Yeah, right!"}).

This situation is actually outside the linguistic rules of semantics and grammar, it's verbal irony, specifically the kind communicated through the use of inflecting the utterance with a sarcastic tone (or dismissive) in order to negate the entire utterance. The same effect could be achieved by saying \textit{``Yeah...''} once only, or even thrice with an accompanying dismissive tone. Likewise saying \textit{``Yea Right!''} in an emphatic tone, remains a positive assertion. Two positives do not make a negative in English, but the effective use of verbal irony can negate anything. Austin's statement remains factual, despite the pragmatically witty remark of Morgenbesser. 

\subsection{Understatement and Litotes}

Hegel negation of negation or sublation

neither can be both true

``this is green'' 

''this is not green'' disguised contrary

``not green'' means the class of all non-green or a particular non-green

blue is non-green

purple is non-green

``not green'' concrete particular example of not green


Dialectic materialism

compromise is negation of the negation

work

play

another direction along a hidden dimension, something which is both work and play?? play at work, work as play???



\section{Tautology}

%``you know it's warm outside when you step outside and it feels warm''

\begin{center}
\textit{``Apparently you can tell a lot about someone's personality from what they're like''}\\ --- Harry Hill
\end{center}

Some might be call this the anti-joke, stating the obvious etc... but they are definitely jokes, they just operate at a lower level of conscious interpretation. A more explicit (but not vulgar) example would be

\textit{``every sixty seconds seconds in Africa, a minute passes''}

%Or possibly:

%\textit{``They say you are what you eat, but I don't remember eating myself''}

Kant calls these analytic proposition

a priori proposition.

The proposition can be divided into predicate and subject, and the predicate is contained within the subject.

The terms (and their definitions) in the statement are sufficient to evaluate the truth-value of the proposition.

A rose is a plant

Is a analytic proposition, as the intension plant is included in the definition of a rose.

A rose is red

would be an example of a synthetic proposition, as the intension red is not included in the definition of rose.

Tautology: 

undeniably true by virtue of it's form.

AND also saying the same thing twice, a redundant repetition of extraneous words.


Gottlob Frege's logical semantics allows the substitution of synonymous terms.

how - tautology functions in our everyday use of language, e.g.. when you say a rose is a rose. why is this a contradiction, "a rose is ..." the formal structure is subject and predicate. you expect something and it's a total surprise when it's the same. red, plant... you get the same. The contradiction of expectation.

Further:

in everyday language we see repetition, "man is a man"

the naive understanding is that this statement is redundant, but we know it's not

the implication that there does exist a man that is not a man

there is the implicit stereotype that one could be a "real man", "true man" or "proper man", and that another is an "improper man", "not a real man"

the expectations to signal a thing that it doesn't fit it's own concept

\newpage
\section{Conditional Definition}

\begin{center}
\textit{``Hitler went to a fortune teller and asked her, `On what day will I die?' The seeress assured him that he would die on a Jewish holiday. `Why are you so sure of that?' demanded Hilter. `Any day,' she replied, `on which you die will be a Jewish holiday.' "}
\\ --- 1930's American Street Joke
\end{center}

This wonderful American street joke entered circulation between 1932 and 1939. It's best appreciated when considering the original context of the joke, that is just before WW2, Hitler is both alive and the leader of Germany, the Holocaust has not happened, nor could anybody even imagine the horror of such a thing. The implication of the joke is that the Jews at the time hated Hitler so much they would celebrate his death with glee. This breaches a social taboo, namely celebrating death of another, specifically a popular politician. Even if the politician was known for antisemitic rhetoric in public speech. The joke no longer carries a heavy punch in 2019, as celebrating Hitler's death is no longer taboo. Satirical jokes do not fix society, they just narrate the zeitgeist.

Linguistically there is an ambiguity lying in ``Jewish Holiday" as referring to an already established holiday, but also a hidden denotation of a day of celebration which will be established. The joke also has a deeper significance on the conditional nature of definition. 

--- lacking a link here


The definition of words are contingent on subjective conditions we impose.

%This is a dangerous game, because if you say that all good people support 

The subjective nature of meaning

We often define x because of y, but this is clearly not true because x could also be y in this case.

The logical fallacy of such definitions is perfectly explained by the esoteric Boston comedian Steven Wright with a joke I heard on a 2015 bootleg recording. It's not in any of his publicly released work yet.

\begin{center}
\textit{
``...and it started snowing out, and I said `oh man I can't believe it, it always snows on my birthday' and he said `so it's your birthday?' I said `nope' [laughter] I just wanted to prove to him the primitive linear thinking by taking those two facts and coming to that conclusion. [more laughter]''} \\ --- Steven Wright
\end{center}

Our intuitive understanding of conditional statements aren't always logical.

Causal conditionals
Indicative conditional
material conditional

They always make the Modus ponens inference, but only half the time do they make the Modus tollens inference even though the information content is the identical. 

when given Counterfactual conditional statements they make both inferences.

It is evident that our cognitive processing of the information content relies heavily on the linguistic structure.

The Psychoanalytic Perspective of conditional inferences is best described by Jacques Lacan and one of his favoured jokes.

``My fiancée is never late for an appointment, because the moment she is late, she is no longer my fiancée.''

``Alphonse Allais's well-known joke, quoted by Lacan: somebody points at a woman and utters a horrified cry, "Look at her, what a shame, under her clothes, she is totally naked" (Lacan, 1986, p.231)''

``I didn't sleep with my wife before we were married... I don't even know what was her maiden name was.''

``First guy: I didn't sleep with my wife before we were married, did you? Second guy: I don't know. What was her maiden name?''

People often define something by what they desire it to be. 

this jokes literally plays on the idea of reinterpreation of semantics in a direct way.
%A Truth is never enforced, because the moment fidelity to Truth functions as an excessive enforcement, we are no longer dealing with a Truth, with fidelity to a Truth-Event

If you love your wife because she is faithful (amongst other good reasons), you do not really truly love your wife, and by true love I mean the irrational and unconditional love of romantic poetry. Because when you outline a reason for your love, you outline a condition which may not always be true, you outline the possibility that you will fall out of love, specifically here when she is unfaithful. The neurochemical emotion of true love is controlled by dopamine addiction, the same feeling that fuels drug addiction. You don't seek your next hit of Meth for a good reason, similarly you don't seek out the company of you beloved for a good reason.

Limerence, or infatuation, or blind love.

Love isn't an economic transaction, that you should choose the best partner for a well-formed logical reason, it's supposed to be irrational. It's the kind of love that leads you to irrational behaviour like sacrificing your own well being for theirs.

A mother loves her child simply because they are her child, that is true love. If you love God simply because he is God then that is also true love. There being no qualifying condition for your love. You love God not because God loves you, not because God made you, not for the promise of Heaven, not for forgiving your sin, but simply because he is God. This sounds like the most irrational logic, and it is, however it is also the only correct way to love God. God will never stop be anything but God.

Although it was Heidegger who joked ``Lacan is the psychiatrist who is himself in need of a psychiatrist''. So what would he know.


\newpage
\section{Illogical Logic}

\begin{center}
\textit{``A young boy discovered that yelling `JUMP!' at a spider causes the spider to jump. The scientifically curious boy then removes the spiders legs and again yells `JUMP!', but the spider remains still... thus concludes the boy ``Spiders with no legs are deaf.''}
\\ --- Street Joke
\end{center}

The joke therein is the obvious faulty reasoning of affirming the consequent. There is actually a second level to the joke, where it is the case that spiders \textit{do} hear with their legs. They have fine hairs called Trichobothria that sense sound waves in a way similar to the hairs within a mammalian cochlea. 

On the fallacy however, the conclusion that \textit{``spiders without legs cannot be ordered to jump''} is true from experimental empirical evidence, but the underlying justification to explain the phenomena is wrong.

The reasoning is not deductive, it's more abductive.

%XKCD 3 x 9
%https://xkcd.com/759/

assuming that causation implies causation.
fallectious reasoning/.

The Empiricist has no problem with the science

There is an observable experiment, which is repeatably provable

%Hume called these \textit{relations of ideas} and not \textit{matters of fact} 

Kant would call this a synthetic proposition, because it's contingent on some excluded information.

Conventional wisdom just says it's correlation not causation.

But here's the catch, the inaccuracy of a model does not preclude it's usefulness. 

% Simulation vs Emulation

Lifting weights build's muscle bulk

Athletes need to give their muscle tissue rest time because...

The process of building muscle mass causes micro tearing in the muscle tissue, so you need rest time to repairs the tearing.

This is faulty reasoning!

Muscle tissue does not actually "tear", it's a far more complex system of signals and hormone responses

But the faulty reasoning leads athletes to give their bodies rest, to give their muscles time to heal the `micro tears'

%The same holds for the statement Lifting weights increases your muscle mass. In the past they used to say that weight lifting caused the “micro-tearing of muscles,” with subsequent healing and increase in size. Today some people discuss hormonal signaling or genes, tomorrow they will discuss something else. But the effect has held forever and will continue to do so.

%Green Lumber * Nicholas Taleb

psychohydraulic

% Relief theory
\subsection{Analogy}

The conventional understanding of Freud is that his research has been ``widely discredited'', and that he is a perverted deviant obsessed repressed libidinal desire... and he totally was, but we should not underestimate his influence on modern understanding of the mind. Most of his good ideas have been subsumed into our collective conscious. When we say things like \textit{``steaming mad''}, or \textit{``let off some steam''}, or when we see a cartoon depiction of anger as steam venting out from the ears, we are referencing the Freudian model of psycho-hydraulic emotion. Freud promoted the idea that emotion was psychic energy, and that if not outwardly vented it would build like the pressure inside a steam boiler. This analogous model of emotion is just plain wrong according to modern neurobiology. 

The faulty reasoning was contrived because an angry outburst appears \textit{explosive}, and it is provably unhealthy to \textit{bottle} emotion. From the modern psychoanalytic perspective these observations are both correct correct, despite the underlying model being wofully outdated, there is plenty of evidence that proves that explicating your own thoughts and even past traumas to another, or into a private journal is the best way to understand your psyche and overcome any trauma. And in a wider context, confronting your demons head on is the only way to overcome them. Do not avoid your problems, they do not disappear, a monster under your bed unconfronted will only remain.

I think Freud's model of emotion is clearly an analogy, as it successfully connects the perceived similarity of two different contexts (steam-boiler $\rightarrow$ emotive-mind) in order to translate the understanding of one onto the other. The muscle micro-tear model is a different kind of analogy, it's markedly different because it connects the same context onto itself, namely the physical behaviour of fictional-muscle, onto the physical behaviour of real-muscle. In this sense it's more like a reflexive analogy. The actual model of anything is a \textit{reflexive-analogy}. For example if I describe the mechanics of a combustion engine through an analogy to the combustion engine. Or possibly through an analogy to the \textit{Verbrennungsmotor}\footnote{German for Combustion Engine} and the elements of my analogy might include:

\begin{center}
\textit{Pistons, Valves, Gasoline, Spark-plugs, A Crankshaft and Explosions}
\end{center}

each which are mapped onto another vocabulary set 

\begin{center}
\textit{Kolben, Ventil, Benzin, Zündkerze, Eine Kurbelwelle and Explosionen}
\end{center}

I would be describing the model as itself. My explanation would be oversimplified and possibly inaccurate as I'm not an automotive engineer, nor can I speak German.


\newpage
\section{Phenomenology}

\begin{center}
\textit{``If a tree falls in the forest and nobody hears, my illegal logging business is a success''}
\\ --- Daniel Crisp
\end{center}

\noindent
The first interpretation is figurative[Metaphysical], the second is literal[Economic].

\begin{center}
\textit{``I saw a tree fall in the woods, and I didn't hear it''}
\\ --- Steven Wright
\end{center}

\noindent
The joke is in the logical contradiction, a paradoxical statement.

\noindent
\begin{center}
\textit{``If a tree falls in the forest and there is no one around to hear it\\ does it still make a sound?''}
\\ --- Classic Forest Proverb
\end{center}

The classic Forest Proverb is either an ambiguous or vague statement. 
It's a joke-in-waiting. Or it's a linguistic riddle ready to be solved by a wannabe philosopher!

%My friend\footnote{who has a PhD in Russian Literature} tells me that one interpreation of understanding is that this is \textit{Defamiliarisation}, as explained by the formalist V. Shklovsky. A more precise translation as a \textit{reflexive-re-contextualisation}, an artistic technique to highlight what society usually ignores, or has become jaded too. This is true.

My friend's\footnote{Who openly claims to have MENSA IQ, without evidence} solution is \textit{``like a Schrodinger-sound... it may or may not exist and isn't defined until we either hear it or don't''}. Now this is an analogy of thought, so there is a punchline in there somewhere.. maybe the cat's death is \textit{instant!} before the sound waves could reach \textit{it's} unlucky ears...? The \textit{Schrodinger's Cat} is an analogy of the wave function collapse. The Copenhagen interpretation of quantum events is the theory that the unobserved event can exist in a superposition of states and reduces to a single state at the moment of observation.

As a linguist, my interpretation of the proverb is about how we define observable phenomena using epistemic language. The unobserved sound does not exist in a superposition of ambiguity, but rather our definition does. The disparity between the objective event and the subjective experience. %GGit just is what is is, irrespective of our epistemic-subjective experience. 
%The standard definition of sound is that it is heard in one way. 
The subjective act of observing the sound doesn't cause the sound-itself to change, but rather hearing a sound causes the subject (or listener) to change, they change into someone who has heard the sound. Maybe there is also a quantum state change; the reflection of subjective experience; the universe looking back at us... but I'll such metaphysical musing to a panpsychist.

\newpage
\subsection{It's a Phenomenon!}
I define the phenomenon of sound, or any perceptible observation to consist of at least 5 distinct parts (or linguistic intensions), they are:

\begin{enumerate}
    \item \textbf{The Object} to generate the signal (tree falling)

    \item \textbf{The Signal} of particles/waves (transverse-waves, aroma-particles, photons)

    \item \textbf{The Medium} to transmit  (air, water, electrical-wire, vacuum)

    \item \textbf{The Transducer} to receive (ear, eye, webcam, antennae)

    \item \textbf{The Subject} to perceive (biological or artificial mind)
\end{enumerate}

This is the transition from the objective world into the subjective world.

The last two intentions are hyponyms of the \textit{subjective listener}.

From this extrapolated definition the ambiguity of the verb-phrase \textit{``to make a sound"} has at least 2 clear definitions.

\begin{enumerate}
    \item \textbf{The Objective} airborne transverse wave; the first 3 Intensions.

    \item \textbf{The Subjective} phenomenon of sound; all 5 Intensions.
\end{enumerate}

I think the forest proverb highlights the ambiguity of the verb-phrase in the English language. We can redefine the objective sound definition through the negation of completeness to be an \textit{incomplete-sound}, which is partially-synonymous with the subjective phenomenon of \textit{silence}.

% What purpose is there for a word that describes a phenomenon that by definition cannot be directly observed? 
% we do have words defined for quantum events that we do not personally observe, we do however indirectly observe them through scientific instruments like CERN etc... whereas we cannot design an instrument to measure an unobserved phenomenon, because the measurement is the act of observation. 

\subsection{Evidentiality}
% The root cause of the proverb is the lack of evidentiality.
How can one consider a fallen tree if we have no evidence that the tree fell in the first place? an unobserved event isn't open to examination. Evidence has many levels of trustworthiness. (\textit{witnessed-account, testament, hearsay, double-hearsay, Scientific Fact, Divine Wisdom...}). Evidentiality is optional in English, but the Turkish grammar rules enforce evidentiality. Just as English enforces tense (I did jump, I am jumping, I will jump), the \textit{when} of an anecdotal-event is almost always expressed in English as relative to the moment of speaking or \textit{time of speech}.
In Turkish this proverb could be articulated:
% From the phrase \textit{"Jim sings beautifully"} one could infer that I have in fact heard him sing, but the phrase is actually ambiguous in the same way as the tree proverb. The sentence does not include whether I was around to hear him sing. I instead could have read a review by a trusted friend.

\begin{center}
\textit{"If I \textbf{saw} tree fall in the woods, and nobody was around to hear it, is it still sound?"}
--- Turkish Scarecrow
\end{center}

Now that statement is not sound... by which I mean it's not \textit{logically sound!} The contradiction is more clear within this nonsense question. You might be able to encode the ambiguity of objective/subjective distinction into a Turkish statement, but you will never be able to enjoy the pleasure of an empty park.

%like all good linguists I'm unsure if I can even speak one language.

\newpage
\section{Ownership and Possesion}

\begin{center}
\textit{"I sold my house this week. I got a pretty good price for it...\\ but it made my landlord mad as hell."} 
\\ --- Garry Shandling
\end{center}

This joke exploits the ambiguity of linguistic posessin. the possessive pronoun.

the difference between possession and ownership in English is often times ambiguous. 

In a linguistic sense, but also has legally, politically and social ramifications. 

Linguistics has identified many different forms of possession indication rules, all of which are optional in English, as shown by the ambiguity of the joke.

Obligatory possession
one's own body not *an own body

Inalienable possession

Inherent and non-inherent possession

Possessable and unpossessable

Maasai, possessable and the unpossessable. 

\textit{tools, houses, family members and money} can be possessed

\textit{wild animals, landscape features and weather phenomena} cannot be possessed. 

\textit{my sister} is grammatically correct but not \textit{my mountain}. 

\textit{the mountain that I have the legal deed for} circumlocution

Ownership and posession distinction are necessary for any economic system 

allows for theft

Certain cultures did not develop a distinction between ownership and possession. 

Native American land and Maori land are key examples where a cross-cultural miscommunication was a problem.

Ownership through a political body

Ownership through community consensus

Ownership through as a personal attitude

A Strict Legal sense, to own property. 

Something intrinsic to your self, to own red hair. - inalienable possession

Childern, subjects, wives - inherent posession

To be dominated by...

A strict legal sense delineates between possession and ownership. , copyright law, piracy, sharing,
to enjoy, occupy, rent, sell, give away even destory an item of property. An item can be material like a house, or immaterial like a copyright, or debt.

If you "own" your iPhone can you destroy it? can you install third-party software on it? can you legally "jailbreak" your "owned" device and install custom firmware? if not, then you do not truly own your iPhone by this definition.

This also asks the question of stealing and theft. Can a joke be owned? can it then be stolen? if you steal a joke, you should probably give it back. 

ownership being a moral right to control something... to prevent it from being taken from you... unless your live in communisim. 

Karl Marx in The German Ideology delineated between different kinds of ownership

Common ownership vs collective ownership

Can one own an idea? you can keep it secret to prevent

patent, copyright law

algorithm copyright

Robert Nozick - Anarchy, State and Utopia

"taxation is theft"

self-ownership, body autonomy

\section{}
"I used to do drugs, I still do, but I used to, too" - Mitch Hedberg

Google - Used - Describing an action or situation that existed for a period in the past.
Merriam Webster - Used to - used to say a situation existed in the past but does not exist now

implies no longer

For what purpose does communicating a state of the past if not to contrast it with now


What's the difference between these statements

\textit{I \textbf{was} doing drugs.}

\textit{I \textbf{used to} do drugs.}

\textit{England \textbf{was} ruled by Queen Victoria.}

\textit{England \textbf{used to be} ruled by Queen Victoria.}

\textit{The Mona Lisa \textbf{was} painted by Leonardo Da Vinci.}

\textit{The Mona Lisa \textbf{used to be} painted by Leonardo Da Vinci.}

The past tense:

The first sentence implies that Queen Victoria no longer rules England.
The second sentence implies that Leonardo no longer paints the Mona Lisa.

The first sentence implies that somebody else now rules England.
The second sentence does not imply that somebody else now paints the Mona Lisa.

The verb \textit{to paint a picture} has a different sort of end point, than \textit{to rule a country} does.


Modal Verb

an auxiliary verb that expresses necessity or possibility. English modal verbs include must, shall, will, should, would, can, could, may, and might.







\section{Unconscious Knowledge}

The Irish alternative stand-up comedian Michael Redmond has a peculiar stage persona, he appears as this anonymous weirdo, he would always wear a long brown mackintosh raincoat, he had a big bushy hair, a droopy moustache and deep-set eyes with a facial expression of forlorn confusion.

\begin{center}
\textit{``I like going into newsagent shops saying, `Excuse me! is that Mars bar for sale?' When says, `Yes,' I say... `OK, I might be back later, I still have a few other ones to see'''}\\ --- Michael Richmond
\end{center}

I struggle to explain this joke by finding a linguistic mistake. But there is some contradicting notion in there, perhaps something deeper within our representation of knowledge and behaviour.

This joke is best described as incongruous, the behaviour described is absurd or silly... it's not the normal behaviour expected when purchasing a cheap chocolate bar, but rather it's the behaviour when you purchase something far more expensive.

bisociation

two implicit frames of references

the purchase of an item


2003

Donald Rumsfeld 

the U.S. Secretary of Defense 

he gave a speech defending the invasion of Iraq 

and surprising he actually said something of value

by induldging in some amateur philosophy

\begin{center}
\textit{``there are known knowns; there are things we know we know. We also know there are known unknowns; that is to say we know there are some things we do not know. But there are also unknown unknowns --- the ones we don't know we don't know.''}\\ --- Donald Rumsfeld
\end{center}

Rumsfeld left the Punnett square incomplete, he left out the unknown knowns, these are the things that we are unaware that we know. These are the unconscious beliefs and prejudices that dictate how we behave and perceive reality through. This is unconscious, implicit or tacit knowledge.

Now I'm sure most people who have bought a car, and bought a chocolate bar were already well aware of these two different modes of economic transactions.

experiment 

five dollar off



\newpage
\section{Situational Irony}

\begin{center}
\textit{``Kant’s Joke. Kant wanted to prove in a way that would dumbfound the common man that the common man was right: that was the secret joke of this soul. He wrote against the scholars in favor of the popular prejudice, but for scholars and not popularly.''}
\\ --- Friedrich Nietzsche
\end{center}

Is there intentional irony in Kant's work? The dead cannot confirm speculation. But there is Situational Irony in a scholar behaving contradictory to their normative behavior. Nietzsche saw the joke, and I laughed at his comment. If only for the glibness of the remark. Treating the work of one of the most lauded philosophers to have ever lived as trivial, it's comedy gold! Nietzsche may very well have been the greatest thinker to have ever lived, if on the condition he considered the entirety of his work as a bizarre thought experiment with no practical value. 
Unfortunately this is not the case and he became a mere comedian. 

\newpage
\section{Cosmic Irony}
%And by doing so you  decompose ethical behaviour into independent rules, which was a novel idea 3,800 years ago.
%at the time\footnote{Which was also invented by Hammurabi at around the same time...}.


\begin{center}
\textit{God commands Abraham to kill his own son...\\ and the madman actually does!}\footnote{Well he does tries to, but at the last possible moment a ram arrives as a replacement sacrifice.}
\end{center}

Cosmic irony is the idea that higher beings amuse themselves by fucking with us cowardly mortals with deliberate or ironic intent, it is sometimes called \textit{irony of fate} or as the Buddhists call it \textit{fate}. The Christian Bible is full of great jokes, The Binding of Isaac is essentially about the contradiction between two modes of ethical behaviour.

\begin{center}
\textit{Do obey God} \\
\textit{Do not kill}
\end{center}

You must explicitly violate one rule to obey the other. This kind of contradiction is shared by all ethical dilemma. The provided solution to this dilemma is to trust God above all else. The Christian apologist argument claims that if the story of deception is true then it's a \textit{test}\footnote{At worst entrapment, at best a cruel practical joke.}. The lesson to be learned is that subservient obedience to God is the paramount ethical maxim, even if it requires blind-faith. This also implicitly gives God permission to lie in an ethical way, which we do not give to mortal-authority figures, as proved by U.S. senator Joseph McCarthy. 

The Islamic apologist argument redefines the son's stance. The son to be murdered in the Muslim account is Ishmael not Issaac, and Ishmael is cooperative and agrees to the sacrifice, so it's not unethical killing if there is consent. %In this account a magical shield of copper appears to protect Ishamel at the last possible moment.

The Judaic apologist argument redefines Abraham's stance. Abraham does not properly commit himself to the action of murder because he has faith that God can see something he himself cannot. Abraham leads the unknowing Isaac to the altar because it's more important how you behave, not what you believe; 

\noindent
compliance despite cognitive dissonance. 

% A similar argument was made by the Christian Philosopher and Father of Existentialism, Kierkegaard\footnote{Fear and Trembling 1843}, who rejected many interpretations of the story and concludes: \textit{``By faith Abraham did not renounce his claim upon Isaac, but by faith he got Isaac"}. Which is wonderfully well put, Abraham did not resign himself to the loss of his son, he was able to obey God \textbf{and} trust that Isaac would be safe, which isn't rational behaviour, but neither is belief in God.
A similar argument is made by the Christian Philosopher and Father of Existentialism, Kierkegaard\footnote{Fear and Trembling 1843}, who rejected many interpretations of the story and concludes: \textit{``By faith Abraham did not renounce his claim upon Isaac, but by faith he got Isaac"}. Which is wonderfully well put, Abraham did not resign himself to the loss of his son, he was able to obey God \textbf{and} trust that Isaac would be safe, which isn't rational behaviour.

\newpage
This kind of logic to explain God's behaviour is reminiscent of the joke Freud used to describe the strange logic of dreams, namely the variety of mutually exclusive solutions to a problem. 

\begin{center}
\textit{``In a dream you are accused of returning a broken kettle to a friend, you defend yourself by saying''}:
\end{center}
\begin{enumerate}
\item \textit{I returned it to you unbroken.}
\item \textit{the kettle was broken before I borrowed it.}
\item \textit{I never borrowed a kettle from you.}
\end{enumerate}

All three solutions are inconsistent with each other, but they all negate the same propositional accusation, and affirm the alternative: \textit{you did not break the kettle}. All arguments defending The Binding of Isaac negates the idea that God coercing Abraham was not essentially unethical.% all explanations being mutually incompatible.

\begin{enumerate}
\item The lesson justifies the means.
\item The proposition is not unethical killing, it is rather consensual killing.
\item The behaviour merely appears unethical, in reality it has only ethical intentions.
%\item Abraham was coerced into behaviour that only \textit{appears} to be unethical.
\end{enumerate}

The psychoanalytic interpretation is that Abraham initially experienced an intrusive thought, and like an impulsive child he could not ignore this thought. Abraham clearly lacked the ability to distinguish the unconscious thinking of a waking-dream from a divine vision, or possibly he just lacked the will-power to not act on the thought. %The abstract concept of free will had yet to be formally articulated by any scholar, likely a foreign concept to all who lived at the time.

Your unconscious (subconscious) mind is constantly forming neural connections, some of these develop into simulations of possible actions in your environment, most of these thoughts sublimate immediately as absurd, illogical, or utter nonsense. However some of the more plausible (or conceivable) pre-cognitive thoughts bubble to the surface and are noticed by your conscious awareness. Only after the thought has ascended through many subconscious levels of increasing complexity, intermingling with other thoughts and evolving itself into a well-formed thought, does it become articulable as: \textit{grab-the-cup}. Both the \textit{awareness of thirst}, and the \textit{awareness of cup} coincidentally coincide within your unconscious, later on you become aware of the thought at which point your conscious reason can agree with your unconscious imaginings and you decide to act upon it, or not. That's the basic theory of unconscious thought. 

Intrusive thoughts are demarcated by their connotations of violent, disturbing, sexual, unacceptable or abhorrent nature. They are typical seen in patients diagnosed with OCD, PTSD, Tourette's, Agoraphobia, Schizophrenia, being an asshole and other forms of mental-illness. The mentally-well also have these thoughts, they're fundamental to creative-insight, lateral-thinking and divergent-thought; it's when your wandering eye accidentally undresses an attractive person, or when you're holding a baby while standing next to an open window, or set of stairs, or a cliff... We are quick to admonish ourselves for these thoughts thinking we might be a terrible person. But it's actually the exact opposite, people who have these thoughts are safer to be around, professional psychoanalysts actually recommend indulging in these thoughts (unless they cause trauma or anxiety in which case seek out professional guidance) since it's just your subconscious mind figuring out a potential danger\footnote{or assessing a potential sexual partner ;)}, so listen! and decide consciously if it's ethical or not.

Essentially Abraham is not mentally ill because he imagined a vision or heard the voice of God, he is because he \textit{acted} on it. Thankfully the miscommunication between his self-conscious mind and his unconscious mind did not lead to his son's death... supposedly.

\newpage
\section{The Deep Joke}

% \begin{center}
%     \textit{``James Watts a neurosurgeon who performed the first frontal lobotomy died this week. If you recall, a lobotomy involves drilling holes in the skull and then inserting and rotating a knife around to destroy brain cells. What a genius, he'll be missed.''}
%     \\ --- Norm Macdonald
% \end{center}

% \begin{center}
%     \textit{``Yippy! Jerry Rubin died last week...\\
%     Oh I'm sorry, that should read: Yippie\footnote{Yippie: Member of the radical anti-establishment group ``Youth International Party''}, Jerry Rubin died last week... Sorry about that, I'm sorry, my mistake completely, I just didn't read it right.''}
%     \\ --- Norm Macdonald
% \end{center}

\begin{center}
    \textit{``The first Miss America, Margaret Gorman died this week at the age of 90. A call-in vote will determine whether she'll be buried in a bathing suit or an evening gown.''}
    \\ --- Norm Macdonald
\end{center}

% \begin{center}
%     \textit{``The actor who appeared in commercials for many years as the Marlboro Man has died of lung cancer. When asked if he contracted the disease by smoking a spokesperson for to tobacco industry responded: `what's that behind you?!' then ran away.''}
%     \\ --- Norm Macdonald
% \end{center}

%If you cannot believe speaking ill of the dead is funny, I encourage you to watch this brief video of Weekend Update: \url{https://www.youtube.com/watch?v=BgTNWJNN6F8}

Many jokes transgress taboo, break a normative custom, or breach a social rule. 
Most social rules are followed tacitly, adopted through nonconsious behavioural mimicry, 
while others must be stated explicitly, especially for those develop-mentally delayed.

\noindent
For any rule to be communicated emphatically it requires a formulation in language.

\begin{center}
\textit{``\textgreek{τὸν τεθνηκóτα μὴ κακολογεῖν}'' --- do not speak ill of the dead}
\\ --- Chilon of Sparta 
\end{center}

In Philisophical Investigation (1953) Wittgenstein described jokes that have the character of depth as \textit{Grammatischer Witz}. The kind of jokes that express deeply rooted uneasiness which arise \textit{``through misinterpretation of our forms of language''}. Wittgenstein saw all variety of social interaction as \textit{Form's of Life} which are governed by the rule's of \textit{Language Games}. His definition of language rules (or speech rules) appear to also include social rules. I was unsurprised to hear that people who knew him have posthumously diagnosed him with high-functioning autism spectrum disorder. He was a stuttering recluse who had a proclivity for yelling, arguing, and temper tantrums, as a man in his 50's he would storm out of rooms if he didn't like what people were saying. His behaviour was clear evidence he lacked the basic social skills most people take for granted. 

\textit{``Do not speak ill of the dead''} is a language rule. This is an ambiguous definition. Does it mean a rule which \textit{governs language usage}? or a rule \textit{defined with language}? or both? It is true that you can indirectly break the \textit{social rule} by directly breaking the \textit{language rule}... however you could instead urinate directly on Hitler's grave.

Is the correct interpretation of Wittgenstein's philosophy that, language encompasses all forms of social interaction? Thankfully, the dead cannot refute speculation, so I'm free to use his corpse as a mouthpiece for my ideas. Hopefully the association of credibility will imbue it with the illusion of validity. This legendary philosopher passed away in 1951, and I like to think he's up in heaven arguing over something pedantic with St. Peter... or maybe he's in hell, where Demons gnaw at his flesh and the agony of the damned will never cease... either way, this extraordinary savant will be missed.

% https://wittgensteinforum.wordpress.com/2007/06/29/did-wittgenstein-have-aspergers-syndrome/

%That's what me just thinked, so I wroted it.









%%%%%%%%%%%%%%%%%%%%%%%%%%%%%%%%%%%%%%%%%%%%%%%%%%%%%%%%%%%%%%%%%%%%%%%%%%%%%%%%%%%%%%%
%%%%%%%%%%%%%%%%%%%%%%%%%%%%%%%%%%%%%%%%%%%%%%%%%%%%%%%%%%%%%%%%%%%%%%%%%%%%%%%%%%%%%%%
%%%%%%%%%%%%%%%%%%%%%%%%%%%%%%%%%% END OF CHAPTER %%%%%%%%%%%%%%%%%%%%%%%%%%%%%%%%%%%%%
%%%%%%%%%%%%%%%%%%%%%%%%%%%%%%%%%%%%%%%%%%%%%%%%%%%%%%%%%%%%%%%%%%%%%%%%%%%%%%%%%%%%%%%
%%%%%%%%%%%%%%%%%%%%%%%%%%%%%%%%%%%%%%%%%%%%%%%%%%%%%%%%%%%%%%%%%%%%%%%%%%%%%%%%%%%%%%%









\newpage
\section{UN FINISHED JOKE IDEAS}
``Boxers don’t have sex before a fight, do you know why that is? Because they don’t fancy each other.'' - Jimmy Carr
elliptical expression

%make an (often normative) assumption or attach a predicate
%punchline is then delivered by going in a different direction or attaching an alternate predicate.
 
``I used to do drugs… I still do, but I used to too''

I come from a very large family—nine parents.

%  “I used to do drugs,” pausing, and then revealing, “I still do, but I used to too”

% Mitch Hedberg says: “I used to do drugs. I still do, but I used to too.”9
% He deliberately violates the Maxim of Manner; he could have conveyed the
% same information more briefly and less obscurely by saying “I’ve been doing
% drugs for a long time.” Instead, he generates the implicature that he no longer
% does drugs, which he then immediately cancels

"Tea or coffee? Yes, please!"

\subsection{Socialism}

``There is an old joke about socialism as the synthesis of the highest achievements of the whole human history to date: from prehistoric societies it took primitivism; from the Ancient world it took slavery; from medieval society brutal domination; from capitalism exploitation; and from socialism the name...'' --- Slavoj Žižek

the meaning of a word
signifier and signified
similar to the word "jew" that antisemitics 

the greed they get from bankers, they filth they get from the poor, and from jews the name..

\subsubsection*{Another fun fact}
Sometimes the similarity between words is contentious, in Canaanite mythology Asheroth and Astarte are gods that can be seen in the sky as astronomical objects, but much like Superman and Clark Kent they never once appeared in the same room together, because they are both the same planet we call Venus. The closest English equivalents are the Morning Star and Evening Star respectively. They are not the same gods but is the ``Morning Star'' synonymous with ``Evening Star''? historically they were distinct, is ``Venus'' interchangeable with either?

It depends on the context or the language game.
Sometimes it does not matter which you refer to
Sometimes it does

They are different concepts, since the occurence of the morning or evening star can be used to determine whether it is before sunrise, or after sunset.

Because jokes rely on false interpretations, they 

Phenomenology and Metaphysics

performative utterance

propositional content

Wittgenstein 

\subsection{truth}
common belief about truth and falsity leads to a contradiction

propositional content can be assigned a truth value

a grammatically correct sentence 

The Liar paradox was solved by noticing that semantics of a language can be bivalent, but the pragmatics of natural languages do not conform strictly to bivalent logic, an utterance can be true or false but also unknown, undecidable, ambiguous, vague….


Wittgenstein

limitations of a perfect system.

\subsection{Derrida}
Deconstructionist and Grammatoligist

Critical look at institutional power structures, metaphysical ideology and manifestations of it
Deconstruct the policy of immigration
Deconstruct rhetoric of nationalism
Deconstruct the politics of place
Deconstruct the metaphysics of native land and native tongue
Disarm the bombs of identity that nation-states build to defend themselves from the strangers, Jews, Arabs, immigrants...

The subjective depiction of the nation of the U.S.A as a power with contemporary influence on our planet
You can see their unification is a single unified discrete entity in American-English language
The use of the phrase “...the united states are… ” was popular but after WWII
It has been replaced with “...The United State is… ”
They are no longer a collective of city states, but are an inseparable entity that is contrasted against the world. Binary opposition. 

You can see the implicit prioritization and self aggrandized-ment of their status in the world by calling their country “America” as a metonym for the continent.

How does an a self identifying “American” see their own Country?
How does an iranian perceived the united states of America
, the country that unjustly went to war against vietnam, iraq, afghanistan, 

When somebody describes themselves as America... they could be meaning the conventional definition of the USA, or the more of the continent of America.

Differance - distinction between speech and phenomena. Word meaning is nebulous , there is no precise and accurate definition for any word, since it depends on other word definitions.
Language is a self-supporting structure

\subsection{more unsorted jokes}

A woman gets into a taxi in Boston’s Logan airport and asks the driver, \textit{``Can you take me someplace where I can get scrod?''} He says, \textit{``Gee, that’s the first time I’ve heard it in the pluperfect subjunctive.''}


\subsection{Rule of Three}

the minimal number of items in-order to establish, and the break a pattern.

Monday, Tuesday... banana.

Napoleon must lose twice to get the point

the first failure could be contigent on bad luck, the second was affirmation.

\subsection{belief in belief}
Daniel Dennett belief in belief, it's not the belief itself wich is beneficial, but the belief in belief.



It is said that a visitor once came to the home of Nobel Prize–winning physicist Niels Bohr and, having noticed a horseshoe hung above the entrance, asked incredulously if the professor believed horseshoes brought good luck. \textit{“No,”} Bohr replied, \textit{“but I am told that they bring luck even to those who do not believe in them.”}

\subsection*{Ellipsis}
you will all die... before I leave this stage.

Infer the last part of the sentence.
Aunty Donna sketch about Mark's internet girlfriend being a man.

Many cases of metonymy are due to elliptical expressions, an intentional omission that the reader is expected to naturally infer. When one says they have been ``reading Shakespeare'', the word ``Shakespeare'' does not refer to the man, but rather his written works, i.e. ``Shakespeare'' $\longrightarrow$ ``Shakespeare's written works''. 

The word ``France'' could refer to the political entity, the country's leaders, or a geographic area. Ellipsis is extremely prevalent in many different areas of natural languages, as we will see later.

Another example would be when we use ``Google'' to refer to ``Google's Web Search Engine'', not the company ``Google'', or the verb meaning ``to search the internet''.