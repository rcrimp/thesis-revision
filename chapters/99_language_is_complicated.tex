\chapter{Language is Intricate}

I like to think that I am a competent speaker, that I am a good writer, that I have a large vocabulary and the skills on how to use it—one of the sharper cookies in the jar. I hope my High-School English teacher would be proud of how far I have come. She spent great time, effort, and care with me, ensuring I obtained NZQA University Entrance, including allowing me to retake the year 12 curriculum while still attending class with my year 13 peers.

Despite the linguistic prowess I now claim to have, I will admit that I still struggle. Grappling with language is much more difficult than I had anticipated. Largely because I misunderstood what language is.

\section{Language is Incomplete}

In Chapter \ref{chap:language}, I said that antonymous pairs are interdependent. One cannot exist without the other. This made sense when I wrote it, especially when considering complementary antonyms (e.g.\ alive \& dead). However, this is not the case, language is incomplete, and there are many concepts for which there does not yet exist an antonym.

``Disambiguate'' does not have an antonymous pair, no verb exists in common usage that means\MarkText[red]{``ambiguate''}, the closest I can think of is ``obfuscate'', but even that is a stretch. Another example is ``underwhelm'';\MarkText[red]{``whelm''} and\MarkText[red]{``overwhelm''} do not exist. It is certainly disappointing. I had hoped it would be\MarkText[blue]{appointing}. One could argue that these words do indeed exist, and if my vocabulary were better, I would know them, and the counterargument I would give is the antonym for fragility. 

\newpage
Fragile things break under pressure. If it does not break under pressure, it is robust, but something which \textit{strengthens} under pressure had no word until recently.

\begin{center}
\textit{``What does not kill me makes me stronger.''}
\\ --- Friedrich Nietzsche, Twilight of the Idols
\end{center}

Nassim Nicholas Taleb coined the word antifragile in 2012 \cite{taleb2012antifragile}. Many concepts that fit the description of antifragility preexisted in the world, Nietzsche's above aphorism; Greek Mythology: the fabled hydra growing two heads for every decapitation; apparent physical anomalies: non-newtonian (Thixotropic) fluid etc. But a word denoting the concept for antifragile did not exist until 2012. Now that the word exists, it is much easier to categorize examples that fit the description.

% Was language complete before the discovery of hydrogen?
% Was language complete before mathematicians 

Not only is language incomplete, but so is our collective understanding of concepts. 1,000 years ago, the concept for \textit{survival of the fittest and death of the weakest} theory of evolution did not exist. Now it is common knowledge. I'm certain our future holds many more ground-breaking conceptual leaps, and when they arrive, so too will new words to describe them.

% Semantic Change
% https://oxfordre.com/linguistics/view/10.1093/acrefore/9780199384655.001.0001/acrefore-9780199384655-e-323
% https://www.uni-due.de/SHE/SHE_Change_Semantic.htm

% Richard Dawkins meme 1976 book The Selfish Gene from (modeled on gene)

% unknown or entirely fabricated words
% using an existing word in a metaphorical manner
% Weakening
% Strengthening
% hypebole
% literally

% Google came from gogle
% 
% The classic 100 words for snow
% terrific (terror), fantastic (of fantasy), tremendous (to tremble), wonderful (inspiring wonder), fabulous (of fabled legend), 
% Some times words are designed strategically for political influence

\section{Language is Unstable}

Language, it seems to me, is an invisible torrent of change. It's a deluge. It's a flood. It's ongoing inundation of alteration,
constantly rushing past us in these minuscule little increments, these tiny little fractions. Every day words fall out of favour and are removed from dictionaries, seemingly forgotten about. Every day, new words are coined by speakers and authors alike. New words begin life as a neologism. If the word is popular, it will propagate from person to person until it becomes approved by lexicographers and codified into dictionaries, immortalised and recognised officially as an actual \textit{``word''}.

Beowulf, written in Old English, contains at least 37 different words for hero \cite{jespersen1919growth}, but since modern English does not need all those synonyms, they have become lost in time's hourglass. Oftentimes, new words are created because there is a need for them. If there is a gap in a person's idiolect, they will use their creativity to fill it. Metaphorization is a common method for creating words, spontaneously employing an existing word in a figurative fashion to describe an alien concept. We can metaphorize almost unthinkingly, such as describing an uncommon colour. 

% or preceding the new use of the word with \textit{``it's kinda like...''}, or adding the suffix `-ify' or `-ize'. Or adding a prefix, recent additions to the English vocabulary include misgendering and ecoanxiety. 

Or maybe the neologism is planned with more intention, perhaps a portmanteau of existing words such as: brunch, mansplain, Brexit, hermaphrodite\footnote{Hermaphrodite was the offspring of Hermes and Aphrodite}, and bootylicious. The concatenation of existing words such as: eyeball, doublespeak, deepfake, or some other mishmash of portmanteau, concatenation, and abbreviation like COVID-19 (\underline{Co}rona \underline{vi}rus \underline{d}isease 20\underline{19}). Perhaps words are forced to change by an administrative entity in an effort to achieve politically correct language (nowadays called \textit{inclusive language}). Github recently renamed the ``master'' branch to ``main''. Email blacklists are now blocklists or denylists. The Obama administration removed all references to ``mental retardation" from their law books in favour of ``intellectual disability''.

\begin{center}
\textit{``If necessity is the mother of invention, then play is its father.''}
\\ --- Steven Johnson, Wonderland
\end{center}

Sometimes a word is created not out of need but rather for fun. Email spam and internet spam were named after the repetitive and annoying nature of the famous Monty Python ``spam'' sketch. Big Bang was originally intended as a sarcastic diminutive to undermine the theory. James Joyce coined the nonsense word quark, probably an intentional misspelling of a German cheese, which physicists adopted for a subatomic particle. In 1982, Gary Larson coined the word thagomizer in a Far Side comic depicting a cavemen professor pointing at the tail spikes of a Stegosaur's saying: \textit{``Now this end is called the thagomizer ... after the late Thag Simmons.''}. At the time, paleontology lacked a word for these spikes, so thagomizer was adopted. Any cursory glance into etymology reveals just how easily language can be influenced on a whim. Onomatopoeic words can arise like zhuzh, janky, and yeet. Cockney rhyming slang words can graduate from their regional dialect such as blowing a raspberry (tart, fart).

% Similar to how Kiwis understate ``The Tasman Sea'' as ``The Ditch''.
% euphemistic language to avoid obscenities, and to appear modest

A prevalent motivation for neologisms in common speech is polite language. To convey respectfulness, we avoid obscenities or mentioning taboo subjects directly, so we instead use innuendo or euphemism. A euphemism is often deployed with air quotes or raised eyebrows to convey a hidden meaning surreptitiously implied. In the early 1900s, to avoid saying homosexual, people would instead say queer. The Irish understate the three decade long ethno-nationalist, fanatic fueled warfare and terrorism as ``The troubles''. There are hundreds and thousands of euphemisms for sexual acts, drug use, bodily fluids, toilet activities, aging, disease and other forms of death. 

Steven Pinker explains the reason we have so many euphemistic expressions is because of the Euphemism Treadmill (a metaphorization of his own devising). Pinker observed that as a euphemism popularity rises, its meaning becomes more apparent and less hidden. It becomes normalized. It begins to directly reference the taboo thing it was designed to avoid, so another word must be coined in its place.

% An interesting example is descriptions of intellectual disability because most of the words begin the life 
% psychiatrists and doctors to describe mental conditions, but slowly over time through casual repurposing of the words, and the medical community developing they become pejoratives.

% obsolete medical classification

% mental deficiency

% idiot IQ of 0–25
% imbecile IQ of 26–50
% moron" (IQ of 51–70)

% dumb (unable to speak)
% lame (limp / paralysis)
% spastic (muscle spasms)
% crazy
% insane
% hysterical
% retarded
% and most recently
% The R-word

% Recently, in the U.S. Rosa's Law was brought in during the Obama administration that  
% removed all references of ``mental retardation" were changed to ``intellectual disability'' in law.

\subsubsection{The Steep Caf{\'e}}
I noticed the euphemism treadmill myself when visiting the Steep Caf{\'e}, which is next to Baldwin Street, the street currently known as the steepest in the world! The barista assumed that I was a \textit{``local''} and charged me the \textit{``local price''} of \$3.50 for a coffee. She then proceeded to describe to me how the \textit{``Asians"} are often confused by which street is the tourist attraction as the caf{\'e} is one block from Baldwin Street. It immediately struck me as odd that she would refer to them as \textit{``Asian''} and let me naturally infer from the context that she meant \textit{``foreign language speaking tourist unable to read the English signs''}. Back in the gold rush of the 1860's here in Otago, gold miners from China were referred to as \textit{``Chinaman''}, then after that went out of fashion, \textit{``Oriental''} was adopted. Since that term has now become unfashionable, \textit{``Asian''} is its replacement. People have again begun to feel weird about referencing people by their ethnic appearance and will soon move onto a different word, possibly \textit{``Eastern''}... and yet still charge them \$5.50 for a coffee, as a kind of foreigner tax.

I have no intent to ever visit the Steep Caf{\'e} again, except only to suggest their staff should probably circumvent the Euphemistic Treadmill and instead use the word \textit{``tourist''}, but they have not yet reopened since the first COVID lockdown.


%%%%%%%%%%%%%%%%%%%%%%%%%%%%%%%%%%%%%%%%%%%%%%%%%%%%%%%%%%%%%%%%%%%%%%%%%%%%%%%%%%%%

% “Military intelligence is a contradiction in terms.” ― Groucho Marx

% “The Moral Majority is neither.” ― bumper sticker
% The examples given above are all political in nature, but there are many reasons why one might describe something different from what it is.

%%%%%%%%%%%%%%%%%%%%%%%%%%%%%%%%%%%%%%%%%%%%%%%%%%%%%%%%%%%%%%%%%%%%%%%%%%%%%%%%%%%%
\section{Language is Inconsistent}
Our conventional understanding of phrases and compound words is that their meaning is a combination of their constituent parts. However, many words do not mean what they seem.

\begin{center}
\textit{``the Holy Roman Empire was neither holy, nor Roman, nor an Empire.''}
\\--- Voltaire
\end{center}

Greenland is mostly ice land, and Iceland is mostly green land. Who named North Korea the ``Democratic People's Republic of Korea''? Would it be more accurate to refer to it as the ``Kleptocratic Tyrant’s Dictatorship of Korea''? Why do we park on a driveway but drive on a parkway?.

\begin{center}
\textit{``Inflammable means flammable? What a country!''}
\\--- Dr Nick, The Simpsons, Season 12 Episode 18
\end{center}


Even more troublesome than words that mean something different from what they seem are words that simultaneously mean what they seem AND mean their own opposite. Auto-antonym or contronym (against name) often arise from statements written in irony, where the literal sense directly opposes the author's intended sense. My favourite example comes from Bugs Bunny when he called Porky Pig a Nimrod, an allusion to a great hunter from the Bible, intended as a sarcastic statement to demean Porky's hunting skills. Not only has nimrod entered the vocabulary of colloquial American English as an insult for an unskilled or inept person, but it has also become dictionary-codified (Oxford, Merriam-Webster, Google Dictionary).

\begin{center}
    \textit{``A towel gets wetter as it \textbf{dries}.''}
\end{center}

The verb \textit{``to dry''} is a contronym, it can mean moisture leaving (evaporation) or moisture entering (saturation). A recent addition to the contronym family is ``\textit{literally}'', a word that can be used for hyperbolic emphasis, as in the example: ``\textit{Now that I'm 40 years old, I'm literally minutes from death}''. Many people would abhor this twisted use of language, especially in this particular instance. However, it is not our responsibility to impose our own rules on how natural language should be used. Even if we tried, we would probably fail catastrophically, as proven by the French, with their L'Académie Française. They would never admit it as an unmitigated failure, only that it is an interminable project.

%%%%%%%%%%%%%%%%%%%%%%%%%%%%%%%%%%%%%%%%%%%%%%%%%%%%%%%%%%%%%%%%%%%%%%%%%%%%%%%%%%%%


\section{Language is Misleading}

The strong form of the Sapir Whorf hypothesis suggests that language influences and limits our thinking \cite{whorf2012language}. The hypothesis is contentious \cite{hussein2012sapir}, nobody has effectively proven or disproven if the the rules of speech influence how we think, but it is undeniable that our thoughts influence our speech. 

The current state of cognitive linguistics believes that we represent a mental schema of concepts from which we derive words and phrases. The theory is that the automatic way we use language can reveal our underlying mental representations that we ourselves are sometimes unaware of \cite{lakoff2008metaphors, lakoff2008women, pinker2007stuff}. Some believe you can judge a person by how they speak and the words they choose to use; whether this is accurate or fair is debatable.

\begin{center}
\textit{``one's true character can be gleaned from how one treats wait staff.''}
\\ --- The Waiter Rule (Conventional Wisdom)
\end{center}

Many racial epithets and ethnic slurs, are meronyms, words that reduce a person down to a body part typical of their ethnicity (\textit{Black, Brown, Darkie, Redneck, Redskin, Round-eye, Slant-eye, Slope-head, Thicklips, Wetback, Yellow, White, Whitey}). These terms exist for every demographic of humans imaginable: age, gender, sex, race, nationality, religion, occupation, deprivation, educational level, socioeconomic status etc. What many of these offending meronyms share is that they are \textit{objectifying}. This process of \textit{objectifying} is assumed to represent an internal mental process that disrespectfully degrades a person from a sentient being with feelings down to a mere object. If you heard someone use this kind of language with sincerity to describe a person, you would be right to avoid them.

%Other metonyms slurs are being contiguous , 

\subsubsection{Student Teacher}
In High-School we had a young and attractive student-teacher for a term. Her classes would sometimes devolve into disorder as many young males would act up to get her attention or provide her with untoward attention. I distinctly remember once in gym class, she attempted to establish dominance by referring to the class of young boys as \textit{``ladies''} and claim she had \textit{``more balls''} than any of them. Was she reinforcing the gender norms that associate dominance with masculinity and submissiveness with femininity? Was she unaware of her own sublimated bigotry? Was her sexist language evidence of genuinely held sexist beliefs? Or was she just being linguistically lazy, merely repeating phrases without consideration of their subtext?

% And most importantly, does any of this even matter?
%%%%%%%%%%%%%%%%%%%%%%%%%%%%%%%%%%%%%%%%%%%%%%%%%%%%%%%%%%%%%%%%%%%%%%%%%%%%%%%%%%%%

\section{Language is Ill-Defined}
% It depends on what the meaning of the word ‘is’ is
% Since these examples aren't usually considered part of clear communication.
% Philosophers of Language have attempted to develop more comprehensive 

There have been many attempts to fix the shortcomings of language, invent new words where none existed, create exceptions for grammar rules and spelling rules, even redefining definition! That's right, something as fundamental as \textit{word definition} recently required redefinition.

% The best way to describe linguistic concept of \textit{Family Resemblance}, is to use examples of biological families in reality, where the linguistic concept gains it's figurative name.

In Chapter \ref{chap:language}, I defined \textit{word definition} as every condition which is necessary and sufficient to uniquely identify it. This has recently become outdated. This obsolete definition can be found in the MIT Press textbook \textit{Linguistics, an Introduction to Language and Communication}; specifically, the 5th edition (2001) pages 235-236, under \textit{The Sense Theory of Meaning}. However, the 6th edition (2010) of the same textbook only mentions necessary and sufficient conditions to discredit it. Specifically, page 444, where it cites cognitive psychologist Eleanor Rosch and her Prototype Theory \cite{rosch1973internal, rosch1975family}. She explains how semantic categories are fuzzy at their boundaries and category membership is inconsistent. The textbook also cites experimental evidence that supports Prototype Theory. So, sometime in the past few decades, our understanding of semantics fundamentally changed because of the groundbreaking work done by Rosch in the 1970s. Around the same time IR researchers were first attempting simple stemming.

The most concrete examples I can provide to discredit our previous definition of definition comes from the world of biology, the Genus–differentia definition. The scientific definition of \textit{avian} or \textit{bird} cannot include the condition \textit{have wings} because it is not necessary, as the New Zealand Moa ---a now extinct flightless bird--- lacks even vestigial wings but is genetically within the avian classification. Nor can the definition of \textit{fish} include \textit{has no legs}, as the Triglidae (or Gurnard) indeed has legs. six legs evidently evolved from the phalanges of its pectoral fins used to aid underwater locomotion on the seafloor. The Genus–differentia definition is composed of two parts, a classic definition of genus, which includes necessary and sufficient conditions, and a defferentia part which catches the exceptions, i.e. birds without wings and fish with legs.

% The easiest path to understanding Prototype Theory is from its source of inspiration, Wittgenstein's Family Resemblance (1952). Semantic category membership as an analogy of a family membership. The members of a semantic category are like family members, each member has attributes that resemble one of their siblings, but distantly related members of the same family can share few attributes, or sometimes even no attributes. 

% Our ability to describe how language works is imprecise, this is clear when exploring the outer edges and darker corners of linguistics. How do you assign semantics for a double-entendre? Can a statement have two simultaneous semantic definitions? How do you assign semantics for a paradox? Can a statement have a super-position of contradictory meanings? or does it belong to a special category of undecidable semantics? There are various theories to answer these in \textit{Philosophy of Language}, but they don't even feature in textbooks on Linguistics. Nowadays, puns and paradoxes are categorised as \textit{rhetoric} and \textit{sophistry} and outside expectations of conventional communication and so ignored by linguistics. Which is fair. I don't even know anybody who could speak a paradox, though I'm sure they'd make for a good conversationalist. Personally, I would never claim to be a good conversationalist, though I do regularly prove it.

% For the Venn diagram of a single category, there is an empty set at the intersection of all category members, no minimal set of conditions, neither necessary nor.

\section{Language is Imprecise}
Our ability to describe how language works is imprecise, this is clear when exploring the outer edges and darker corners of linguistics. How do you assign semantics for a double-entendre? Can a statement have two simultaneous semantic definitions? How do you assign semantics for a paradox? Can a statement have a super-position of contradictory meanings? or does it belong to a special category of undecidable semantics? There are various theories to answer these in \textit{Philosophy of Language}, but they don't even feature in textbooks on Linguistics. Nowadays, puns and paradoxes are categorised as \textit{rhetoric} and \textit{sophistry} and outside expectations of conventional communication and so ignored by linguistics. Which is fair. I do not even know anybody who could speak a paradox, though I'm sure they'd make for a good conversationalist. Personally, I would never claim to be a good conversationalist, though I do regularly prove it.

% \subsubsection{Wittgenstein}
% Ludwig Wittgenstein was an Austrian born philosopher (1889 - 1951) whose thought is often regarded as the most influential since Immanuel Kant. Described by Bertrand Russell as \textit{``the most perfect example I have ever known of genius}". His second book Philosophical Investigations (posthumously published in 1952), was surveyed to be the most important philosophical work of the 20th century, \textit{``the one crossover masterpiece in twentieth-century philosophy, appealing across diverse specializations and philosophical orientations"} --- \textit{``If you ask philosophers (English speaking) who is the most important philosopher of the twentieth century, they will most likely name Ludwig Wittgenstein"}. And in his final book, he introduced the concept of a language game, and I must admit that I have been playing this game at various levels throughout this entire thesis. 

% Wittgenstein considered all philosophical problems as mere misunderstandings between the speech of two philosophers. Thus, he said, \textit{``there are no real philosophical problems''} only \textit{``linguistic puzzles''}. The root cause of all misunderstandings arises from the unintentional obfuscations that are inherent within language. Philosophy from this perspective is the process of redefining concepts, reclarifying statements, and constant arguments over which linguistic definition is better than another. Nobody doubts how the imprecision of language can cause endless misunderstandings, confusion and frustration, but Wittgenstein's objectors argue that his perspective trivialises philosophy. Karl Popper famously argued against Wittgenstein, proposing that many philosophical problems \textit{``transcend language''}. 

% I should clarify that as I am not the arbiter of truth. I cannot say which perspective is better. I am not necessarily a proponent of Wittgenstein's or Popper's philosophy. I'm not advocating for anything at all. I'm not trying to convert anyone to this way of thinking. I have nothing to sell. I'm an author --- similarly, Douglass Hofstadter's \textit{G{\"o}del, Escher, Bach: An Eternal Golden Braid} has nothing to sell you either, except the beauty of art. Douglass doesn't want to convert you to anything. He doesn't want you to join an organisation that favours Bach music over Beethoven, Escher over Picasso, or G{\"o}del over Gauss. And I write in the same spirit. I just want you to enjoy a point of view that I enjoy.









% \footnote{The clothes of humility are several sizes too small to fit my stature.}

%Because all philiosophical problems like jokes rely on mistakes in language, and our attempt to solve them with language will only lead to more puzzles to be solved. And jokes that are no longer funny.

% Wittgenstein argued with him for a mere 10 minutes until eventually storming out of a room in anger, frustrated that Karl did not understand him.

% I should first clarify before going any further that as I am not the arbiter of truth. I cannot say which perspective is better. I am not necessarily a proponent of Wittgenstein's or Popper's philosophy. I'm not advocating for anything at all. I'm not trying to convert anyone to this way of thinking. I have nothing to sell. I'm an author --- similarly, Douglass Hofstadter's \textit{G{\"o}del, Escher, Bach: An Eternal Golden Braid} has nothing to sell you either, except the beauty of art. Douglass doesn't want to convert you to anything, he doesn't want you to join an organisation that favors Bach music over Beethoven, Escher over Picasso, or G{\"o}del over Gauss. And I write in the same spirit. I just want you to enjoy a point of view that I enjoy.

%Because all philiosophical problems like jokes rely on mistakes in language, and our attempt to solve them with language will only lead to more puzzles to be solved. And jokes that are no longer funny.


% Philosophers of Language have attempted to solve the language problems at the fringes of spoken language, like how to define semantics for a double-entendre, or whether a paradox even has semantics. Can a statement have two simultaneous semantic definitions, or an an undecidable definition? Since these examples aren't usually considered part of clear communication.









% https://en.wikiversity.org/wiki/Dominant_group/Genus_differentia_definition
%%%%%%%%%%%%%%%%%%%%%%%%%%%%%%%%%%%%%%%%%%%%%%%%%%%%%%%%%%%%%%%%%%%%%%%%%%%%%%%%%%%%
\section{Language is Subjective}

I was taught how to speak as a child by adults much older than me, they were wiser than me, they knew everything about anything. Whenever I misspoke, they corrected me, and I assumed that their corrective judgement referenced some source of truth, an objectively defined language that one can learn. I thought this objective language had been captured in books, codified in dictionaries and grammar textbooks, but I was wrong. I know this because if I offer my family members a ``camel'', my uncle and niece will get excited at the proposition because my uncle is a smoker, and my niece loves animals. Language is subjective; we all learn the word ``camel" through usage in context. All of language is learned from our own unique experiences. Language doesn't come from dictionaries as they aren't a source of truth. They are stamp collections, outdated and obsolete before they've been printed. Nobody reads a dictionary to expand their vocabulary; people only read the dictionary to settle scrabble disagreements.

% Language started long before grammar textbooks existed

% grammar is an attempt to ensconce language into a set of rules

% one can observe rules and declare them

% but one cannot invent rules and enforce them

% any attempts to enforce rules fails

% ending a sentence with a preposition

% etc...

% there is always an exception to rules

I was taught that if a misunderstanding occurs one should blame either the speaker or the listener — the speaker's inability to communicate with clarity or the listener's inability to listen properly. I never thought language itself caused trouble and strife. Due to the subjective nature of language, from within our minds, the language we use has the illusion of being fit for communication. However, when viewed from a distance, it becomes clear that language is, inconsistent, imprecise, and incomplete. Trying to remove these errors from language using language: is like trying to remove salt from the ocean using a sieve made of salt.

% Language is an unstable self-supporting structure of sand, and you cannot restructure the sand beneath you to make the sand above you more stable.

Before I started this thesis, I incorrectly assumed language as a tool was complete. During this thesis, I incorrectly assumed linguistics as a field was complete. I wonder if any linguists consider search engine technology complete, or maybe I'm the only na{\"i}ve idiot in academia.

% \section{Alternative vocabulary}
% One could blame the complexity of language with it's vast size, however if we look at languages with constrained vocabularies we find that ambiguity persists as a problem, and sometimes becomes worse.

% Randall Munroe proposes a vocabulary in \textit{Thing Explainer}, which consists of the English languages \textit{``ten hundred words people use the most often''}.

% LogBan
% programming languages are known to contain ambiguities, where there are situations of undefined behaviour where different compilers / interpreters behave differently. 

% distinguishing what is a meaningful sentence, from one that is flawed is a non-trivial problem. In fact proving any grammar as unambiguous is an undecidable problem. Which means you must analyze every possible sentence of a language to exclude the possibility of mistakes. And since spoken languages like English are infinitely recursive, it would take an impossible amount of time to find all possible linguistic errors.



%%%%%%%%%%%%%%%%%%%%%%%%%%%%%%%%%%%%%%%%%%%%%%%%%%%%%%%%%%%%%%%%%%%%%%%%
%%%%%%%%%%%%%%%%%%%%%%%%%%%%%%%%%%%%%%%%%%%%%%%%%%%%%%%%%%%%%%%%%%%%%%%%
%%%%%%%%%%%%%%%%%%%%%%%%%%%%%%%%%%%%%%%%%%%%%%%%%%%%%%%%%%%%%%%%%%%%%%%%



\section{Language is Not Thought}
In the introduction of this thesis, when discussing semantics, I claimed that \textit{words} and \textit{meaning} (Signs and Signified) were tightly interconnected like the two sides of a M{\"o}bius strip. Thought however, is an entirely different entity. Thoughts pre-exist language. Babies can think without language, animals can think without language, and ancestral humans all once thought without language.

\newpage
Our thoughts can be difficult to express accurately. In contrast, our feelings are much easier to express. We can express our feelings in any number of ways, with a dance, a poem, a song, a single tear cascading down a rosy cheek in a dimly lit, wintery park. However, the only way we can express our thoughts are in words. We all know thoughts do not sit easily in words. We've all had a thought in our head at some point and considered it an intellectual insight, a majestic marvel, but as soon as we speak that thought, we watch it topple to the floor like an errant ice cream cone.

Thought and language are two very different mediums. Thought is a capricious, flighty, transient bumble-bee. Ever fluctuating, ever-changing, smoke on the wind, sunlight through rain, sand through your fingers. Language, in contrast, is a flat-footed monkey playing with mud in a dirty bucket. Trying to capture the nuance and the complexity of thought into the lumpen stone prose of language is like trying to convince that monkey to put his bucket down, wash his hands, pick up a fishing rod, and catch the wind.

Even the most talented writers can struggle to capture the complexity of thought into language because when one writes, one can only ever say a single thing at a time. That's all any of us can ever do, but that's not how any of us think. Nobody thinks one thought at a time. No matter how seemingly simple, every single thought we have may be surrounded by an endless spider diagram of other interconnected thoughts spiralling off and happening almost simultaneously. Sub-clauses, caveats, counter-arguments, non sequiturs\footnote{In the first draft, I misspelled ``non sequitur'' so badly my spellchecker suggested ``non-secateurs''. I briefly imagined how difficult it would be to trim a hedge with non-secateurs and concluded it would be impossible as they would be \textit{non}-secateurs}. This means if we try and lift any thought out of that spiralling nexus and render it in language, it becomes facile. It becomes clumsy. It becomes redundant. Because it is no longer one of those many things our mind tripped upon in that complex nanosecond of thought, it becomes rendered in language. It becomes \textit{``what we think''}, and it becomes accessible by other people to be misunderstood.

Truth exists, it is real, it is certain, and precise, but the very moment you attempt to distil truth into language, something is lost in the process. Language is an insufficient tool to capture the observations of our reality. Language at its best can only ever hope to resemble the shadows of truth, and at a distance, the silhouette may appear authentic. But when investigated closer, it becomes obvious that it is empty of substance.

% The best we have to communicate information is language. Trying to fix information systems with language. 

% Trying to remove inconsistencies from language using language itself, is like trying to remove salt from soup using a sieve made of salt.

\newpage
To answer the implicit thesis question: \textit{``Can we remove the language barrier that separates us from machines?''} --- No, we cannot even remove the language barrier that separates you and me, or one from themselves, because the barrier is made of language. A language barrier is not a static, structured, unmoving brick wall but is instead a chaotic, unkempt, entangled, bramblous vine, which grows ceaselessly into unfathomable circuitous complexities that intermingle with itself into a multiplicity of unexpected self-intersecting impossibilities, and we cannot stop it. It is, however, a noble task to ensure these vines are arranged more orderly than they were before. We can structure the underlying trellis to encourage growth conducive to the clarity of communication; growth that allows the transmission of information, knowledge, wisdom, and truth across the grapevine and into the future.

\bigskip

\begin{center}
    Thank you very much for reading my thesis, I hope you enjoyed it.
\end{center}


%%%%%%%%%%%%%%%%%%%%%%%%%%%%%%%%%%%%%%%%%%%%%%%%%%%%%%%%%%%%%%%%%%%%%%%%%%%%%%%%%%%%

















% Modern IR systems disambiguate the query and disambiguate the corpus simultaneously. A difficult task that could be simplified by separating the two disambiguation tasks. While there is a common language between the authors of the corpus and the authors of search queries, direct term matching doesn't account for differences in dialect, vocabulary, prose or style. Authors can come from different educational backgrounds, different cultures, different centuries even.

% \textit{The most important thing in communication is to hear what isn’t being said.} Peter F. Drucker

% \textit{How true it is that words are but the vague shadows of the volumes we mean. Little audible links, they are, chaining together great inaudible feelings and purposes.} Theodore Dreiser

% \textit{For none of us can ever give the exact measure of his needs or his thoughts or his sorrows, and language is a cracked kettle on which we beat out tunes for bears to dance to, while all the time we move the stars to pity.} Gustave Flaubert

% \textit{Each one of us is alone in the world. He is shut in a tower of brass, and can communicate with his fellows only by signs, and the signs have no common value, so that their sense is vague and uncertain. We seek pitifully to convey to others the treasures of our heart, but they have not the power to accept them, and so we go lonely, side by side but not together, unable to know our fellows and unknown by them. } W. Somerset Maugham






% Trying to distill thought into language is like trying to capture sunlight in a jar.

% Chasing objective truth is like chasing the setting sun towards the earth's edge, we can discover the illumination of new landscapes of wisdom, but we will not discover an end to the journey. 


% Trying to explain language is like trying to describe the contours of a flame.
% Trying to explain language using language itself is like trying to trace the outline of a pencil using the pencil. You can approximate it, you can trace it's shadow, but it won't be a perfect reproduction.
% No amount of words, not even a thousand, can truly capture an image.







% The academic fields of scholarship attempt to distill observations into facts, but there is no direct association, language sits between our material world of phenomena and the immaterial world of meaning.

%Facts are made of language.

% \textbf{this section needs work - i.e. account for nonsense words without phenomena e.g. ``""}

% Meaning exists only within your mind, it is the bisociation between language and phenomena. Words do not have inherent meaning, nor does phenomena have inherent meaning, but rather we can attach a meaning that we simultaneously attach to a particular phenomena. The meaning exists interdependently with both the word and the phenomena. But what about invented words like " 

% Meaning is the subjective interpretations of phenomena and words that is contained within our minds.

% Wittgenstein believed that we can only infer word meaning from how the word is used, which is the perspective of the linguistic descriptivist. 

%Using language to disprove fallacies formed with language 

% Philosophy will never be finished before we can distill objective truth outside of language. That doesn't mean that the pursuit of objective truth is a wasted effort, in this regard I disagree with Wittgenstein. Solving ethical dilemma is one of the crucial roles philosophers hold within humanity, to make order from the chaos, even if much of the chaos is caused by language itself.


% I'm not saying thought can't exist without language, of course it can, every night my conscious mind halts and my unconscious mind hallucinates a myriad of thought without language. I dream in images, in feelings, in sensations, in ideas, but hardly a single word arises. The moment I awake, my inner monologue engages, it attempts translate the narrative of the dream into articulated knowledge, which acts as a foundation that retains the memory, allowing for traversal, and recollection, as if it were any other memory. More often however, the dream sublimates into a translucent haze before the attachment of any language hooks.



%Ludwig Wittgenstein was an Austrian born philosopher (1889 - 1951) whose thought is often regarded as the most influential since Immanuel Kant. Described by Bertrand Russell as``the most perfect example I have ever known of genius". His second book Philosophical Investigations was surveyed to be the most important philosophical work of the 20th century, ``the one crossover masterpiece in twentieth-century philosophy, appealing across diverse specializations and philosophical orientations" --- ``If you ask philosophers (English speaking) who is the most important philosopher of the twentieth century, they will most likely name Ludwig Wittgenstein". And in his final book he introduced the concept of a language game, and I must admit that throughout this entire thesis I have been playing this game at various levels.

% At the beginning of my thesis I quoted the greatest philosopher of the 20\textsuperscript{th} century:

% Wittgenstein remarked to his friend Norman Malcolm that a serious work in philosophy could consist entirely of jokes

% \begin{center}
% \textit{``A serious and good philosophical work could be written consisting entirely of jokes" }
% \\---  Ludwig Josef Johann Wittgenstein 
% \end{center}

% What he is saying is that both jokes and philisophical problems expressed in language rely on linguistic mistakes. Now hopefully the previous chapter should have convinced you that all jokes rely on an intentional exploitation of a linguistic mistake for \textit{comedic effect}, however I will forgive you for not immediately acknowledging that all philosophical problems rely on some linguistic mistake. Although Wittgenstion did believe this to be true, 

% he considered all philosophical problems merely as misunderstandings between the speech of two philosophers. Thus \textit{``there are no real philosophical problems''} only \textit{``linguistic puzzles''}. And the root cause of all misunderstandings arise from the unintentional obfuscations that are inherent within language. 

% Philosophy from this perspective is the process of redefining concepts, reclarifying statements, and constant arguments over which linguistic definition is better than another. Nobody doubts how the imprecision of language causes endless misunderstandings, confusion and frustration, but Wittgenstein's objectors argue that his perspective trivialises philosophy. Karl Popper famously argued against Wittgenstein proposing that many philosophical problems \textit{``transcend language''}. Wittgenstein argued with him for a mere 10 minutes until eventually storming out of a room in anger, frustrated that Karl did not understand him.

% I should first clarify before going any further that as I am not the arbiter of truth. I cannot say which perspective is better. I am not necessarily a proponent of Wittgenstein's or Popper's philosophy. I'm not advocating for anything at all. I'm not trying to convert anyone to this way of thinking. I have nothing to sell. I'm an author --- similarly, Douglass Hofstadter's \textit{G{\"o}del, Escher, Bach: An Eternal Golden Braid} has nothing to sell you either, except the beauty of art. Douglass doesn't want to convert you to anything, he doesn't want you to join an organisation that favors Bach music over Beethoven, Escher over Picasso, or G{\"o}del over Gauss. And I write in the same spirit. I just want you to enjoy a point of view that I enjoy.

%Because all philiosophical problems like jokes rely on mistakes in language, and our attempt to solve them with language will only lead to more puzzles to be solved. And jokes that are no longer funny.

% \section*{Radical Vagueness}
% The extreme variant of this problem is semantic holism (Quine 1960), Quine suggested that individual word have no inherent meaning, which was an idea he ironically expressed using individual words. Wittgenstein's ladder expresses a similarly paradoxical notion: a ladder constructed of words, that once ascended proves that \textit{all} words (of this form) are meaningless, so you must then throw the ladder away, which is ill advised if you're afraid of metaphorical heights. Such radical views on language don't really help us in any practical sense, so we'll leave the metaphysical musings and linguistic acrobatics to those who willingly choose to languish within language. In the following section we too shall willingly make that choice.

% 20 years before she first published Prototype Theory the posthumous publication of Ludwig Wittgenstein's work describes Family Resemblance

% was an Austrian born philosopher (1889 - 1951) whose thought is often regarded as the most influential since Immanuel Kant. Described by Bertrand Russell as``the most perfect example I have ever known of genius". His second book Philosophical Investigations was surveyed to be the most important philosophical work of the 20th century, ``the one crossover masterpiece in twentieth-century philosophy, appealing across diverse specializations and philosophical orientations" --- ``If you ask philosophers (English speaking) who is the most important philosopher of the twentieth century, they will most likely name Ludwig Wittgenstein". And in his final book he introduced the concept of a language game, and I must admit that throughout this entire thesis I have been playing this game at various levels.

% Dr.\ Stephen Jay Gould spent a lifetime studying fish and concluded that there's no such thing as fish. Which isn't like the time researchers tried to study men who've never watched porn and discovered they couldn't find a single eligible candidate. Gould did manage to find what we call ``fish'' but discovered that because of genetic diversity of ``fish'', there cannot be an encompassing definition for fish. The venn diagram of all fish contains no overlap. Salmon is more closely related to a camel than it is to a hagfish. The disparity is between the unseen genes, and the seen phenotypes. 

% ---- Evolutionary convergence
% Carcinisation - convergent evolution of crabs

% Phenotypes can be similar but share not genetic history, both butterflies and birds have wings and eyes, but both features evolved independently. Their similarity is merely evolutionary convergence of form, their similarities are analogous. Whereas the wings of bats to birds are homologous, and the wings of butter flies and bees are too homologous.
